\documentclass{beamer}
\usepackage[brazil]{babel}
\usepackage[latin1]{inputenc}
\usepackage[T1]{fontenc}
\usepackage{lmodern}

\date{20 e 21 de Agosto de 2010}
\usetheme{Goettingen}

\begin{document}

\title{Instalação e Gerenciamento de Pacotes no Slackware}

\author[Nilton Moura]{por Nilton Moura}

\institute{\\
Apresentado no:\\
Instituto Infnet\\
no VI Slackware Show\\
\url{http://slackshow.slackwarebrasil.org/}}

\begin{frame}
	\titlepage
\end{frame}

\section{Breve História}
\begin{frame}
	\frametitle{Breve História do Slackware}
	\begin{itemize}
		\item Patrick Volkerding precisava de um interpretador LISP para um
			projeto. Começou então a aprender Linux com uma das raras
			distribuições existentes, o \textit{SLS Linux}.
		~\\
		~\\
		%\pause
		\item Conforme utilizava, Patrick corrigia os bugs encontrados, porém
			o mantenedor do SLS Linux (Peter MacDonald) não aceitou as
			correções, então Patrick disponibilizou-as no ftp de sua
			universidade.
		~\\
		~\\
		%\pause
		\item Com o passar do tempo o número de correções cresceu, ganhando
			popularidade rapidamente. Patrick então lança para todo o mundo
			em 17 de julho de 1993 o seu trabalho na versão 1.0 com o nome de
			Slackware.
	\end{itemize}
\end{frame}

\section{Filosofia}
\begin{frame}
	\frametitle{Filosofia do Slackware}
		\begin{itemize}
			\item O Slackware é a distribuição mais antiga em pleno
				desenvolvimento até hoje, e tem a reputação de ser a mais
				\textit{UNIX-like}. É a mais tradicional distribuição
				linux.
		\end{itemize}
		%\pause
		\begin{block}{Simplicidade e estabilidade}
			\begin{itemize}
				\item Não tenta ser um clone ou parecer com o Windows
				\item Não tenta fantasiar/encobrir processos ~--- coloca o usuário
					no controle permitindo-o que veja exatamente o que acontece
				\item Configuração transparente ~--- arquivos de configuração bem
					comentados
				\item Cada aplicação é configurada de acordo com a intenção do
					desenvolvedor da aplicação ~--- O Slackware não tem um arquivo
					ou ferramenta de configuração "global"
			\end{itemize}
		\end{block}
\end{frame}
\begin{frame}
	\frametitle{Filosofia do Slackware (cont.)}
		\begin{block}{Simplicidade e estabilidade (cont.)}
			\begin{itemize}
				\item Modelo de desenvolvimento conservador ~--- os programas são
					muito bem testados antes de entrarem ou substituirem outros,
					mas o Slackware não utiliza versões antigas de software
			\end{itemize}
		\end{block}
		%\pause
		~\\
		Slackware é para pessoas que gostam de aprender e ajustar seu sistema
		para fazer exatamente o que eles querem. É por este motivo que as
		pessoas o utilizam por vários anos, tanto para servidores robustos
		quanto para seus desktops pessoais.
\end{frame}

\section{Desenvolvedores}
\begin{frame}
	\frametitle{Desenvolvedores do Slackware}
	\footnotesize{
	Patrick Volkerding ~--- volkerdi\@@ \\
	Eric Hameleers ~--- alien\@@ \\
	Piter PUNK ~--- piterpunk\@@ \\
	Robby Workman ~--- rworkman\@@ \\
	Alan Hicks ~--- alan\@@ \\
	Amritpal Bath ~--- amrit\@@ \\
	Erik Jan Tromp ~--- alphageek\@@ \\
	Stuart Winter ~--- mozes\@@ \\
	Karl Magnus Kolst\o ~--- karlmag\@@ \\
	Leopold Midha ~--- netrixtardis\@@ \\
	Mark Post ~--- markkp\@@ \\
	John Jenkins ~--- mrgoblin\@@ \\
	Vincent Batts ~--- vbatts\@@ \\
	Fred Emmott ~--- fred\@@ \\
	}
	~\\
	Existem outros desenvolvedores que preferem o anonimato por causa de seus
	cargos em seus empregos ou algum outro motivo.	
\end{frame}

\begin{frame}
	\frametitle{Desenvolvedores do Slackware (cont.)}
	\textbf{A comunidade de usuários} ~--- É ela quem contribui principalmente
	com testes e reporte de bugs
	~\\
	~\\
	%\pause
	O que seria do Linux sem a comunidade de usuários?
\end{frame}

\section{Gerenciamento de Pacotes}
\begin{frame}
	\frametitle{Gerenciamento de Pacotes}
	Um mito se fez quando a Red Hat lançou o RPM, que dizia que Slackware
	não tinha gerenciamento de pacotes: a verdade é que têm mesmo antes
	do Red Hat existir.
	~\\
	%\pause
	\center{Gerenciamento de \textbf{pacotes} $\neq$ Checagem de \textbf{dependências}}
	~\\
	~\\
	%\pause
	\begin{block}{Utilitários}
		\begin{itemize}
			\item \texttt{pkgtool}
			\item \texttt{installpkg}
			\item \texttt{upgradepkg}
			\item \texttt{removepkg}
			\item \texttt{slackpkg}
		\end{itemize}
	\end{block}
\end{frame}

\begin{frame}
	\frametitle{Utilitários para Gerenciamento de Pacotes}
	Um pacote é um arquivo \textit{tar} comprimido com algum algoritmo de compressão
	como gzip ou LZMA (xz) por exemplo.\\
	%\pause

	\begin{block}{Como funciona a instalação de um pacote?}
	\begin{enumerate}
		%\pause
		\item O pacote é descompactado e extraído para o diretório raiz (/),
			ou para outro, caso a variável \texttt{\$ROOT} for
			setada.
		%\pause
		\item Se o pacote tiver o script de pós-instalação \textit{install/doinst.sh},
			ele será executado.
		%\pause
		\item É criado um arquivo de log em /var/log/packages/ com o nome 
			\small{\textit{NOME-VERSAO-ARCH-BUILD}}, com várias informações sobre o
			pacote, incluindo a lista de arquivos instalados. Se o script
			de pós-instalação \textit{doinst.sh} existir, ele será copiado
			para /var/log/scripts com o mesmo nome do log.
	\end{enumerate}
	\end{block}
\end{frame}

\begin{frame}
	\frametitle{Arquivos de um pacote}
	\scriptsize{
	\# \texttt{explodepkg btpd-0.15-x86\_64-1\_SBo.tgz\\
	%\pause
	Exploding package btpd-0.15-x86\_64-1\_SBo.tgz in current directory:\\
	./\\
	install/\\
	install/slack-desc\\
	usr/\\
	usr/doc/\\
	usr/doc/btpd-0.15/\\
	usr/doc/btpd-0.15/CHANGES\\
	usr/doc/btpd-0.15/README\\
	usr/doc/btpd-0.15/btpd.SlackBuild\\
	usr/doc/btpd-0.15/COPYRIGHT\\
	usr/bin/\\
	usr/bin/btinfo\\
	usr/bin/btpd\\
	usr/bin/btcli\\
	}}
	%\pause
	\begin{block}{\bf{doinst.sh}}
	O script \textit{doinst.sh} é utilizado para criar links simbólicos e fazer
	comparações de arquivos pré-existentes (que podem ser customizados pelo
	administrador do sistema) com arquivos que serão copiados, com auxílio do
	\texttt{md5sum}. Normalmente arquivos de configuração e scripts de inicialização.
	\end{block}
\end{frame}

\begin{frame}
	\frametitle{Exemplo de doinst.sh}
	\scriptsize{
	\tiny{
	config() \{\\
  	~~NEW="\$1"\\
	~~OLD="\$(dirname \$NEW)/\$(basename \$NEW .new)"\\
	~~\# If there's no config file by that name, mv it over:\\
	~~if [ ! -r \$OLD ]; then\\
	~~~~mv \$NEW \$OLD\\
	~~elif [ "\$(cat \$OLD | md5sum)" = "\$(cat \$NEW | md5sum)" ]; then\\
        ~~~~\# toss the redundant copy\\
    	~~~~rm \$NEW\\
  	~~fi\\
  	~~\# Otherwise, we leave the .new copy for the admin to consider...\\
\}\\
~\\
	\# Keep same perms on rc.mysqld.new:\\
	if [ -e etc/rc.d/rc.mysqld ]; then\\
	~~cp -a etc/rc.d/rc.mysqld etc/rc.d/rc.mysqld.new.incoming\\
	~~cat etc/rc.d/rc.mysqld.new > etc/rc.d/rc.mysqld.new.incoming\\
	~~mv etc/rc.d/rc.mysqld.new.incoming etc/rc.d/rc.mysqld.new\\
	fi\\
~\\
	config etc/rc.d/rc.mysqld.new\\
~\\
	( cd usr/lib ; rm -rf libmysqlclient\_r.so.16 )\\
	( cd usr/lib ; ln -sf mysql/libmysqlclient\_r.so.16 libmysqlclient\_r.so.16 )\\
	( cd usr/lib/mysql ; rm -rf libmysqlclient\_r.so.16 )\\
	( cd usr/lib/mysql ; ln -sf libmysqlclient\_r.so.16.0.0 libmysqlclient\_r.so.16 )\\
	}}
\end{frame}

\begin{frame}
	\frametitle{slackpkg}
	Ferramenta que automatiza o gerenciamento de pacotes criado pelo desenvolvedor
	do Slackware Piter PUNK. Excelente para instalar ou atualizar pacotes pela
	rede.\\
	%\pause
	~\\
	\begin{block}{Algumas características}
		\begin{itemize}
			\item Busca arquivos específicos
			\item Remove pacotes de terceiros
			\item Instala novos pacotes
			\item Atualiza pacotes instalados
			\item etc.
		\end{itemize}
	\end{block}
\end{frame}

\begin{frame}
	\frametitle{Exemplo de uso do slackpkg}
	Como manter seu sistema atualizado com slackpkg:\\
	~\\
	\begin{enumerate}
		%\pause
		\item \texttt{\# slackpkg update}
		%\pause
		\item \texttt{\# slackpkg upgrade slackpkg}
		%\pause
		\item \texttt{\# slackpkg install-new}
		%\pause
		\item \texttt{\# slackpkg upgrade-all}
		%\pause
		\item \texttt{\# slackpkg clean-system}
	~\\
	~\\
	Mais informações em \url{http://slackpkg.org/} ou "\texttt{man slackpkg}".
	\end{enumerate}
\end{frame}

\section{Instalando outros softwares}
\begin{frame}
	\frametitle{Instalando outros softwares}
	O Slackware vem com um conjunto de pacotes grande e variado, mas você pode
	precisar de um software que não esteja incorporado à distribuição.
	%\pause
	~\\
	\begin{block}{O que fazer?}
		%\pause
		\begin{itemize}
			\item Procurar no Google o pacote compilado?
		%\pause
			\item \texttt{./configure \&\& make \&\& make install}?
		\end{itemize}
	\end{block}
	%\pause
	Além da segurança de um sistema, a forma como é instalado um software pode dar
	muitas dores de cabeça. Compilar e instalar manualmente funciona, mas a manutenção
	se torna cada vez mais difícil e trabalhosa. Problemas de compatibilidade com
	algum pacote e dificuldade para remoção são alguns.
\end{frame}

\begin{frame}
	\frametitle{Fontes confiáveis}
	Você pode baixar um pacote de sites cujo criador do pacote tenha boa
	reputação na comunidade.\\
	~\\
	Alguns sites com pacotes compilados confiáveis:
	~\\
	\begin{itemize}
		\tiny{
		\item Eric Hameleers (alienBOB) ~--- \url{http://slackware.com/~alien/}
		\item Robby Workman (rworkman) ~--- \url{http://rlworkman.net/pkgs/}
		\item Erik Jan Tromp (alphageek) ~--- \url{http://alphageek.dyndns.org/}
		\item Niels Horn ~--- \url{http://www.nielshorn.net/slackware/}}
	\end{itemize}
	~\\
	Existem outros sites que reúnem pacotes de diversos autores, que também tem bons
	pacotes, mas nem sempre se tem garantia de qualidade. É bom checar a reputação,
	o próprio pacote, etc.
\end{frame}

\begin{frame}
	\frametitle{SlackBuilds.org}
	O projeto SlackBuilds.org é um repositório de uma grande coleção de scripts
	SlackBuild escritos em sua maioria pela comunidade de usuários, fundado
	e mantido por alguns membros do time de desenvolvimento do Slackware.\\
	%\pause
	~\\
	O que é um script SlackBuild?\\
	~\\
	%\pause
	É um shell script que automatiza o processo de configuração, compilação,
	e criação do pacote a que ele foi preparado.
	~\\
	Site do projeto: \url{http://www.slackbuilds.org/}
\end{frame}

\begin{frame}
	\frametitle{sbopkg}
	O \textit{sbopkg} é uma ferramenta que sincroniza com o repositório
	SlackBuilds.org, que tem diversas funcionalidades, como por exemplo,
	instalar diversos softwares em apenas uma linha de comando.\\
	~\\
	Confira o projeto criado por Chess Griffin em: \url{http://www.sbopkg.org/}
\end{frame}

\section{Créditos e Agradecimentos}

\begin{frame}
	\frametitle{Créditos e Agradecimentos}
	\begin{itemize}
		\item A Deus acima de tudo
		~\\
		~\\
		\item Ao Robby Workman por permitir que eu usasse sua
			apresentação como base, principalmente em História e Filosofia:
			\url{http://rlworkman.net/slackshowbrasil/}
		~\\
		~\\
		\item Ao Alan Hicks e outros contribuintes do SlackBook que também foi
			base para esta apresentação.
		~\\
		~\\
		\item Aos que fizeram o V SlackShow acontecer, a minha gratidão
			pela oportunidade de estar aqui.
	\end{itemize}
\end{frame}
\end{document}
